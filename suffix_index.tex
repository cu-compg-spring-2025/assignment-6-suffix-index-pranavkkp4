\documentclass{article}
\usepackage[utf8]{inputenc}
\usepackage{listings}
\usepackage{color}
\usepackage{hyperref}
\usepackage{graphicx}

\definecolor{mygreen}{rgb}{0,0.6,0}
\definecolor{mygray}{rgb}{0.5,0.5,0.5}
\definecolor{mymauve}{rgb}{0.58,0,0.82}

\lstset{ %
  backgroundcolor=\color{white},   % choose the background color
  basicstyle=\footnotesize\ttfamily,        % size of fonts used for the code
  breaklines=true,                 % automatic line breaking only at whitespace
  captionpos=b,                    % sets the caption-position to bottom
  commentstyle=\color{mygreen},    % comment style
  escapeinside={\%*}{*)},          % if you want to add LaTeX within your code
  keywordstyle=\color{blue},       % keyword style
  stringstyle=\color{mymauve},     % string literal style
}

\title{Suffix Index Experiments}
\author{Your Name}
\date{\today}

\begin{document}

\maketitle

\section{Introduction}
This document summarizes the experiments conducted with suffix data structures for aligning reads to a reference. We implemented and tested various methods including the suffix tree, suffix array, and suffix trie with their respective search functionalities.

\section{Suffix Tree}
The suffix tree was constructed using the \texttt{add\_suffix} function which iteratively integrates every suffix of the input string into a tree structure. The suffix tree is then used to build a suffix array.

\subsection{Suffix Tree Implementation}
\begin{lstlisting}[language=Python, caption=add_suffix and build_suffix_tree]
def add_suffix(nodes, suf):
    n = 0
    i = 0
    while i < len(suf):
        b = suf[i]
        children = nodes[n][CHILDREN]
        if b not in children:
            n2 = len(nodes)
            nodes.append([suf[i:], {}])
            nodes[n][CHILDREN][b] = n2
            return
        else:
            n2 = children[b]
        sub2 = nodes[n2][SUB]
        j = 0
        while j < len(sub2) and i + j < len(suf) and suf[i + j] == sub2[j]:
            j += 1
        if j < len(sub2):
            n3 = n2
            n2 = len(nodes)
            nodes.append([sub2[:j], {sub2[j]: n3}])
            nodes[n3][SUB] = sub2[j:]
            nodes[n][CHILDREN][b] = n2
        i += j
        n = n2

def build_suffix_tree(text):
    text += "$"
    nodes = [['', {}]]
    for i in range(len(text)):
        add_suffix(nodes, text[i:])
    return nodes
\end{lstlisting}

\subsection{Suffix Tree Search}
The \texttt{search\_tree} function traverses the suffix tree following a pattern \( P \) and returns whether the pattern is found.

\begin{lstlisting}[language=Python, caption=search_tree]
def search_tree(suffix_tree, P):
    n = 0
    i = 0
    while i < len(P):
        b = P[i]
        children = suffix_tree[n][CHILDREN]
        if b not in children:
            return False  # Pattern not found
        n2 = children[b]
        sub2 = suffix_tree[n2][SUB]
        j = 0
        while j < len(sub2) and i + j < len(P) and P[i + j] == sub2[j]:
            j += 1
        if j < len(sub2):
            return False  # Mismatch occurred
        i += j
        n = n2
    return True  # Pattern found
\end{lstlisting}

\section{Suffix Array}
The suffix array is built by performing a depth-first search (DFS) on the suffix tree to collect the starting indices of all suffixes. This array is then sorted and can be used to perform binary search for query patterns.

\subsection{Suffix Array Implementation}
\begin{lstlisting}[language=Python, caption=build_suffix_array]
def build_suffix_array(T):
    tree = build_suffix_tree(T)
    suffix_array = []
    def dfs(node_idx, depth):
        node = tree[node_idx]
        if not node[CHILDREN]:  # Leaf node
            suffix_array.append(depth - len(node[SUB]))
        for child_idx in sorted(node[CHILDREN].values()):
            dfs(child_idx, depth + len(tree[child_idx][SUB]))
    dfs(0, 0)
    return sorted(suffix_array)
\end{lstlisting}

\subsection{Suffix Array Search}
\begin{lstlisting}[language=Python, caption=search_array]
def search_array(suffix_array, q):
    lo, hi = 0, len(suffix_array)
    while lo < hi:
        mid = (lo + hi) // 2
        if suffix_array[mid] < q:
            lo = mid + 1
        else:
            hi = mid
    return lo
\end{lstlisting}

\section{Suffix Trie}
The suffix trie is implemented as a nested dictionary structure, where every suffix of the string is added into the trie. A special terminal marker (e.g., \texttt{\#}) is used to indicate the end of a suffix.

\subsection{Suffix Trie Implementation}
\begin{lstlisting}[language=Python, caption=build_suffix_trie and search_trie]
def build_suffix_trie(s):
    trie = {}
    for i in range(len(s)):
        current = trie
        for char in s[i:]:
            if char not in current:
                current[char] = {}
            current = current[char]
        current['#'] = True
    return trie

def search_trie(trie, pattern):
    current = trie
    match_length = 0
    for char in pattern:
        if char in current:
            match_length += 1
            current = current[char]
        else:
            break
    return match_length
\end{lstlisting}

\section{Experiments and Results}
The experiments were conducted using various query sequences on a reference string. The key observations include:
\begin{itemize}
    \item The suffix tree was successfully built and allowed for accurate search of complete patterns.
    \item The suffix array, derived from the suffix tree, supported efficient binary search.
    \item The suffix trie was effective in determining the match length for a given query.
\end{itemize}

\section{Conclusion}
The implementation and experiments demonstrate the utility of suffix-based data structures in aligning reads against a reference. Future work could focus on optimizing these structures for larger datasets and exploring additional query functionalities.

\end{document}
